\documentclass{seuthesis-2019}

\title{海军战术概论}{现代化海战}
\studentnum{12345678}
\author{梅时雨}
\department{乌克兰皇室}
\major{海军理论}
\supervisor{黑海舰队司令}
\period{2022.4.16--2022.4.18}


\begin{document}
\maketitle
\makedeclaration
\frontmatter
\begin{abstract}[zh]
摘要内容独立于正文而存在,是论文内容高度概括的简要陈述,应准确、具体、完整地概括论文的主要信息,内容包括研究目的、方法、过程、成果、结论及主要创新之处等,不含图表,不加注释,具有独立性和完整性,一般为400字左右。

“摘要”用三号黑体加粗居中,“摘”与“要”之间空4个半角空格。摘要正文内容用小四号宋体,固定1.5倍行距。

论文的关键词是反映毕业设计(论文)主题内容的名词,一般为3-5个,排在摘要正文部分下方。关键词与摘要之间空一行。关键词之间用逗号分开,最后一个关键词后不加标点符号。
\keywords{关键词1,关键词2,关键词3,关键词4}
\end{abstract}

\begin{abstract}[en]
英文摘要应与中文摘相对应,250个实词左右。采用第三人称介绍该学位论文内容,叙述的基本时态为一般现在时,确实需要强调过去的事情或者已经完成的行为才使用过去时、完成时等其他时态。

ABSTRACT为三号Times New Roman加粗居中。

英文摘要正文为小四号Times New Roman,固定1.5倍行距。英文关键词“KEY WORDS”大写,其后的关键词第一个字母大写,关键词之间用半角逗号隔开。
\keywords{Keywords1,Keywords2,Keywords3,Keywords4}
\end{abstract}
\tableofcontents
\mainmatter
\chapter{绪论}
\section{课题背景和意义}
绪论部分主要论述选题的意义、国内外研究现状以及本文主要研究的内容、研究思路以及内容安排等。

章标题为三号黑体加粗居中;一级节标题(如,2.1 本文研究内容):四号黑体居左;二级节标题(如,2.1.1 实验方法):小四号宋体居左。

正文部分为小四号宋体,行间距1.5倍行距,首行缩进2个字符。

\end{document}